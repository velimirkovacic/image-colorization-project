Razne su tehnike i metode bojenja crno-bijelih slika nastajale tijekom godina. Tradicionalne metode bojenja oslanjale su se na ručni unos i različite algoritme za propagaciju boje. Jedan je primjer \textit{Scribble-Based Colorization}\cite{li2018overview}, gdje korisnik dodaje boje na određene dijelove slike, a algoritam zatim te boje širi po slici.

S razvojem dubokog učenja, došlo je do značajnog napretka u automatizaciji procesa bojenja. Rana istraživanja koristila su konvolucijske neuronske mreže (CNN), pri čemu je mreža učila mapirati crno-bijele slike u obojene slike na temelju velikog skupa obojenih slika. Jedan primjer te metode predstavlja istraživanje \textit{Colorful Image Colorization}\cite{zhang2016colorful} iz 2016., u kojem su autori koristili konvolucijsku neuronsku mrežu za predikciju distribucije boja za svaki piksel crno-bijele slike. Pojavom GAN-ova i cGAN-ova kvaliteta generiranih obojenih slika ponovno je znatno poboljšana. Rad \textit{Image-to-Image Translation with Conditional Adversarial Networks}\cite{isola2018imagetoimage} iz 2017. ilustrira korištenje cGAN-a za različite zadatke translacije sa slike na sliku, uključujući i bojanje crno-bijelih slika.

Nedavno uvedene nadogradnje uključuju korištenje \textit{U-Net} arhitekture neuronske mreže u generatoru, koja omogućuje bolje dohvaćanje konteksta i detalja u slici dodavanjem preskočnih veza (engl. \textit{skip connections}) između slojeva kodera i dekodera. To dovodi do kvalitetnijih rezultata, posebno u zadacima gdje je očuvanje strukturalnih detalja ključno.

Ovaj projekt koristi cGAN arhitekturu s U-Net generatorom i konvolucijskom mrežom kao diskriminatorom kako bi postigao visoku kvalitetu obojenih slika. Ovaj pristup omogućuje iskorištavanje prednosti suvremenih tehnika dubokog učenja za postizanje vrhunskih rezultata u bojanju crno-bijelih slika.
