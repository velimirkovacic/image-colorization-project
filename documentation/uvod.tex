Bojenje crno-bijelih slika zadatak je u području računalnog vida, a glavna mu je primjena restauracija povijesnih fotografija kako bi njihov prikaz bio što vjerniji stvarnosti. Tradicionalne metode bojenja slika iziskuju značajnu količinu ručnog rada, stručnosti i vremena. S razvojem dubokog učenja i konvolucijskih neuronskih mreža, postalo je moguće automatizirano bojenje slika, što smanjuje potrebu za ručnim radom i omogućuje bržu i učinkovitiju obradu slika. U ovom radu bit će opisano korištenje generativnih modela, posebno generativnih suparničkih mreži (engl. \textit{Generative Adversarial Networks} - GAN), za bojanje crno-bijelih slika. Ta metoda uključuje dva modela: generator, koji stvara obojene slike iz nasumičnog vektora, i diskriminator, koji pokušava razlikovati generirane slike od stvarnih. Specifično je opisano korištenje uvjetnog GAN-a (engl. \textit{conditional} GAN - cGAN), koji na ulazu diskriminatora i generatora dodaje crno-bijelu reprezentaciju slike kao uvjetni vektor. Ovakav pristup omogućuje generiranje visoko kvalitetnih obojenih slika koje su u većini slučajeva vrlo slične onim stvarnima.